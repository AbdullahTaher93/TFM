
\textsf{Experimental results 1 of CrossOver step:}\\
    \colorbox{blue!30}{\textsf{     No. of chromosomes is: 16}}\\
    \colorbox{blue!30}{\textsf{     CrossOver Operator: Multi Point}}\\
    \colorbox{blue!30}{\textsf{     Keeps best parent to next Population: true}}
    \begin{table}[H]
        \centering
        
        \begin{tabular}{{l l }}\hline
            Parent 1 &CJIBEVGYRTZWUMNSPFQOKHLDAX\\ \hline
            Parent 2 &CJYWORQPDMSNFEXGBTVIZLKUAH\\ \hline
            Parent 3 &GSCYDZXKNHTBMJEOPQLAIFWRUV\\ \hline
            Parent 4 &NTDLUYPVISGXBFMOEKZAHWJRQC\\ \hline
            Parent 5 &DMPBJXAGOSWEVUTNYHRCFZLKQI\\ \hline
            Parent 6   &VYDELMZQJTGSIRUNXAPFWKHOBC\\ \hline
            Parent 7 &FQRZKVHOCNEWGDPUMISLBTXYJA\\ \hline
            Parent 8 &RANHSXEPVUQDGTYBWKLZFJIOMC\\ \hline
          Child 1  &CJYWEVOQRTDSUMNGPFBIKHZLAX\\ \hline
          Child 2  &CJIVORGYDMWUFENSBTPQZLKXAH\\ \hline
          Child 3  &GSTLDZYXNHBOMJEKPQAWIFRCUV\\ \hline
          Child 4  &NTGDUYZXISMJBFOPEKLAHWRVQC\\ \hline
          Child 5   &DMELJXTGOSRNVUAPYHWKFZBCQI\\ \hline
          Child 6   &VYDPLMGOJTSEIRUNXAHFWKZQBC\\ \hline
          Child 7   &FQRHKVSXCNEPGDUYMIWLBTZOJA\\ \hline
          Child 8   &RAQZSXHOVUNEGTDPWKILFJBYMC\\ \hline
\end{tabular}

\caption{Experimental results 1 CrossOver Step}
\end{table}


