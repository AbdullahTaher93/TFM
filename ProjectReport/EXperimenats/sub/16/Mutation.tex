\textsf{Experimental results 1 of Mutation step:}\\
    \colorbox{blue!30}{\textsf{     No. of chromosomes is: 16}}\\
    \colorbox{blue!30}{\textsf{     CrossOver Operator: Mutli Point}}\\
    \colorbox{blue!30}{\textsf{     Keeps best parent to next Population: true}}\\
    \colorbox{blue!30}{\textsf{     Two random Postions(2 bits of chromosome): 26 , 22}}\\
    \begin{table}[H]
        \centering
        \begin{tabular}{{ l l }}\hline
            \multicolumn{2}{c}{Final Population} \\ \hline
           
            chromosome 1 &CJIBEVGYRTZWUMNSPFQOKXLDAH\\ \hline
            chromosome 2 &CJYWORQPDMSNFEXGBTVIZHKUAL\\ \hline
            chromosome 3 &GSCYDZXKNHTBMJEOPQLAIVWRUF\\ \hline
            chromosome 4 & NTDLUYPVISGXBFMOEKZAHCJRQW\\ \hline
            chromosome 5 & DMPBJXAGOSWEVUTNYHRCFILKQZ\\ \hline
            chromosome 6 & VYDELMZQJTGSIRUNXAPFWCHOBK\\ \hline
            chromosome 7 & FQRZKVHOCNEWGDPUMISLBAXYJT\\ \hline
            chromosome 8 & RANHSXEPVUQDGTYBWKLZFCIOMJ\\ \hline
            chromosome 9 & CJYWEVOQRTDSUMNGPFBIKXZLAH\\ \hline
            chromosome 10 & CJIVORGYDMWUFENSBTPQZHKXAL\\ \hline
            chromosome 11 & GSTLDZYXNHBOMJEKPQAWIVRCUF\\ \hline
            chromosome 12 & NTGDUYZXISMJBFOPEKLAHCRVQW\\ \hline
            chromosome 13 & DMELJXTGOSRNVUAPYHWKFIBCQZ\\ \hline
            chromosome 14 & VYDPLMGOJTSEIRUNXAHFWCZQBK\\ \hline
            chromosome 15 &  FQRHKVSXCNEPGDUYMIWLBAZOJT\\ \hline
            chromosome 16 & RAQZSXHOVUNEGTDPWKILFCBYMJ\\ \hline
            
\end{tabular}
\caption{Experimental results 1 Mutation Step}
\end{table}


