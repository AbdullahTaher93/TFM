\chapter{Experimentation}
In this chapter, we will write all the issues related to the experimentation phase of the study, from the first decision-making to the last decision-making to validate the solutions. Throughout the following paragraphs, Arguments will be presented in defense of the decisions to be made and a sufficiently detailed description of the experiment will be presented so that the results can be easily interpreted.in this chapter we will get experimental results for each step of the application from population steps to mutation steps with giving the details of each step And make comparisons of the experimental results, These experiments are divided into two parts that depend on the cipher type.
so, we have two stages of the experimentation, which are:

\section{experimentation of Transposition Cipher}
In this part of the work, we will make experiments for the working steps performed to each step of GAs of Transposition cipher.
\subsection{experimentation of population step}
In this part of the work, we will make experiments for the working steps performed to each step of GAs of Transposition cipher.
in this section, we will make an experiment to generate a set of chromosomes(individuals) which represent the first population,  in this case, each chromosome has a set of numbers which represent DNA Which can not be repeated in another chromosome, for each problem, there is different chromosome type that we can define the chromosome (also sometimes called a genotype) is a set of parameters which define a proposed solution to the problem that the genetic algorithm is trying to solve. The set of all solutions is known as the population. The chromosome is often represented as a binary string, although a wide variety of other data structures are also used \cite{IVGeneticAlgorithm}.\\
\textsf{Experimental results 1 of Population step:}\\
    \colorbox{blue!30}{\textsf{     No. of chromosomes is: 12}}\\
    \colorbox{blue!30}{\textsf{     length of chromosome is: 6}}

\begin{table}[h!]
\centering
\begin{tabular}{l l}\hline
    Key(chromosome 0)&6 4 1 2 5 3\\ \hline 
    Key(chromosome 1)&3 5 1 2 4 6 \\ \hline 
    Key(chromosome 2)&1 5 2 4 6 3 \\ \hline 
    Key(chromosome 3)&1 4 3 2 6 5 \\ \hline 
    Key(chromosome 4)&3 5 2 4 1 6 \\ \hline 
    Key(chromosome 5)&4 3 5 1 6 2 \\ \hline 
    Key(chromosome 6)&6 4 3 2 5 1 \\ \hline 
    Key(chromosome 7)&3 1 5 4 6 2 \\ \hline 
    Key(chromosome 8)&3 6 2 5 4 1 \\ \hline 
    Key(chromosome 9)&5 3 2 4 1 6 \\ \hline 
    Key(chromosome 10)&4 1 5 3 2 6 \\ \hline 
    Key(chromosome 11)&1 3 6 4 2 5 \\ \hline  
\end{tabular}
\caption{Experimental results 1 Population step}

\end{table}



\textsf{Experimental results 2 of Population step:}\\
    \colorbox{blue!30}{\textsf{     No. of chromosomes is: 20}}\\
    \colorbox{blue!30}{\textsf{     length of chromosome is: 10}}\\

    \begin{table}[h!]
        \centering
        \begin{tabular}{{ l l }}\hline
            Key(chromosome 0)&9 10 1 4 6 7 5 2 8 3 \\ \hline
            Key(chromosome 1)&1 7 6 3 9 4 5 10 8 2 \\ \hline
            Key(chromosome 2)&6 4 9 10 3 2 8 1 5 7 \\ \hline
            Key(chromosome 3)&10 8 4 9 2 6 5 7 1 3 \\ \hline
            Key(chromosome 4)&7 3 1 9 8 5 2 4 10 6 \\ \hline
            Key(chromosome 5)&3 7 10 4 5 6 9 2 8 1 \\ \hline
            Key(chromosome 6)&2 8 1 3 6 7 9 10 4 5 \\ \hline
            Key(chromosome 7)&6 1 5 10 3 2 8 4 7 9 \\ \hline
            Key(chromosome 8)&2 4 8 6 7 10 5 1 3 9 \\ \hline
            Key(chromosome 9)&6 8 2 1 7 10 9 3 4 5 \\ \hline
            Key(chromosome 10)&10 7 8 3 9 1 4 5 6 2 \\ \hline
            Key(chromosome 11)&2 1 10 9 8 3 6 5 7 4 \\ \hline
            Key(chromosome 12)&10 3 7 6 9 4 1 2 5 8 \\ \hline
            Key(chromosome 13)&8 1 3 2 6 5 7 4 10 9 \\ \hline
            Key(chromosome 14)&9 8 6 1 10 4 2 3 7 5 \\ \hline
            Key(chromosome 15)&3 7 10 2 6 4 8 5 9 1 \\ \hline
            Key(chromosome 16)&9 5 2 10 7 4 6 1 3 8 \\ \hline
            Key(chromosome 17)&4 10 9 8 1 5 6 3 2 7 \\ \hline
            Key(chromosome 18)&4 2 5 3 1 10 7 6 9 8 \\ \hline
            Key(chromosome 19)&1 10 8 4 7 5 3 9 2 6 \\ \hline
\end{tabular}
\caption{Experimental results 2 of Population step}
\end{table}



\newpage
\subsection{experimentation of Transposition Cipher step}
in this section, we will try to break the ciphertext by using the decoding Transposition Algorithm to get the experimental results which will be:
\begin{tcolorbox}[breakable,notitle,boxrule=0pt,colback=blue!20,colframe=blue!20]
    {
    \textsf{Experimental results 1 of Substitution Cipher step:}\\
    \textsf{     No. of chromosomes is: 16}\\
    \textsf{     CipherText: pnTsarotoiisepnhiCXXrXXX}\\
    \textsf{     PlainText: TranspositionCipher}
    }
    \end{tcolorbox}
\begin{table}[H]
\centering
\begin{tabular}{l l}\hline
    PlainTex0 &YBSQOQZGYVXOVBGUZIBD\\ \hline
PlainTex1 &NXVMKMLWNCBKCXWILJXQ\\ \hline
PlainTex2 &TYCXJXSITGBJGYIFSOYD\\ \hline
PlainTex3 &QRCOFOSNQMEFMRNDSYRI\\ \hline
PlainTex4 &XFZPLPVQXKGLKFQCVBFH\\ \hline
PlainTex5 &PJVFAFNUPRLARJUONQJD\\ \hline
PlainTex6 &LVRZKZBYLTSKTVYOBPVN\\ \hline
PlainTex7 &RODCHCNTRUJHUOTKNGOP\\ \hline
PlainTex8 &YWFNHNMKYGQHGWKCMDWE\\ \hline
PlainTex9 &VDCITIGXVEOTEDXUGPDY\\ \hline
PlainTex10 &LNGAQAOELJDQJNEROKNC\\ \hline
PlainTex11 &DRBYQYGHDKAQKRHOGZRL\\ \hline
PlainTex12 &NROCMCDUNGVMGRUSDZRY\\ \hline
PlainTex13 &ZINAKAOMZFUKFIMROVID\\ \hline
PlainTex14 &SCFLILUPSDKIDCPYUOCR\\ \hline
PlainTex15 &FAXLCLMKFPQCPAKNMDAU\\ \hline
\end{tabular}
\caption{Experimental results 1 Substitution Cipher}

\end{table}



\begin{tcolorbox}[breakable,notitle,boxrule=0pt,colback=blue!20,colframe=blue!20]
    {
    \textsf{Experimental results 1 of Substitution Cipher step:}\\
    \textsf{     No. of chromosomes is: 16}\\
    \textsf{     CipherText: pnTsarotoiisepnhiCXXrXXX}\\
    \textsf{     PlainText: TranspositionCipher}
    }
    \end{tcolorbox}
\begin{table}[H]
\centering
\begin{tabular}{l l}\hline
    PlainTex0 &YBSQOQZGYVXOVBGUZIBD\\ \hline
PlainTex1 &NXVMKMLWNCBKCXWILJXQ\\ \hline
PlainTex2 &TYCXJXSITGBJGYIFSOYD\\ \hline
PlainTex3 &QRCOFOSNQMEFMRNDSYRI\\ \hline
PlainTex4 &XFZPLPVQXKGLKFQCVBFH\\ \hline
PlainTex5 &PJVFAFNUPRLARJUONQJD\\ \hline
PlainTex6 &LVRZKZBYLTSKTVYOBPVN\\ \hline
PlainTex7 &RODCHCNTRUJHUOTKNGOP\\ \hline
PlainTex8 &YWFNHNMKYGQHGWKCMDWE\\ \hline
PlainTex9 &VDCITIGXVEOTEDXUGPDY\\ \hline
PlainTex10 &LNGAQAOELJDQJNEROKNC\\ \hline
PlainTex11 &DRBYQYGHDKAQKRHOGZRL\\ \hline
PlainTex12 &NROCMCDUNGVMGRUSDZRY\\ \hline
PlainTex13 &ZINAKAOMZFUKFIMROVID\\ \hline
PlainTex14 &SCFLILUPSDKIDCPYUOCR\\ \hline
PlainTex15 &FAXLCLMKFPQCPAKNMDAU\\ \hline
\end{tabular}
\caption{Experimental results 1 Substitution Cipher}

\end{table}



\newpage
\subsection{experimentation of Fitness step}
now, we will get the preliminary experimental results to fitness step which will be:\\
\\\textsf{Experimental results 1 of Fitness step:}\\
    \colorbox{blue!30}{\textsf{     No. of chromosomes is: 12}}\\
    \colorbox{blue!30}{\textsf{     length of chromosome is: 6}}\\
    \colorbox{blue!30}{\textsf{     CipherText: pnTsarotoiisepnhiCXXrXXX}}\\
    \colorbox{blue!30}{\textsf{     PlainText: TranspositionCipher}}

\begin{table}[h!]
\centering
\begin{tabular}{l l}
    \hline
    \cellcolor[gray]{0.9} PlainTexs& \cellcolor[gray]{0.9} Fitness\\ \hline
    TsrnapoistionhCpierXXXXX &(15.273331515399999)\\ \hline
    TspanroioitsnheipCrXXXXX &(14.738331975400001) \\ \hline
    pTrsnaoositienChpiXrXXXX &(13.836665915400001) \\ \hline
    psTnraoiotsiehnpCiXXrXXX &(13.0183331654) \\ \hline
    aTpsnriooitsinehpCXrXXXX &(14.4149989054) \\ \hline
    srnpTaistooihCpeniXXXXrX &(14.438332145399999)\\ \hline
    rsTnapsiotioChnpieXXrXXX &(13.4733328354) \\ \hline
    nrpsTatsoioipCehniXXXXrX &(13.5466661254) \\ \hline
    rTpasnsooiitCneihpXrXXXX &(13.2499996654) \\ \hline
    aTnspriotiosinpheCXrXXXX &(15.598331425400001) \\ \hline
    naspTrtiioospihenCXXXXrX &(14.964998475400002)\\ \hline
    pansrToitisoeiphCnXXXXXr &(14.9399984454) \\ \hline
\end{tabular}
\caption{Experimental results 1 of Fitness step}

\end{table}



\\\textsf{Experimental results 1 of Fitness step:}\\
    \colorbox{blue!30}{\textsf{     No. of chromosomes is: 12}}\\
    \colorbox{blue!30}{\textsf{     length of chromosome is: 6}}\\
    \colorbox{blue!30}{\textsf{     CipherText: pnTsarotoiisepnhiCXXrXXX}}\\
    \colorbox{blue!30}{\textsf{     PlainText: TranspositionCipher}}

\begin{table}[h!]
\centering
\begin{tabular}{l l}
    \hline
    \cellcolor[gray]{0.9} PlainTexs& \cellcolor[gray]{0.9} Fitness\\ \hline
    TsrnapoistionhCpierXXXXX &(15.273331515399999)\\ \hline
    TspanroioitsnheipCrXXXXX &(14.738331975400001) \\ \hline
    pTrsnaoositienChpiXrXXXX &(13.836665915400001) \\ \hline
    psTnraoiotsiehnpCiXXrXXX &(13.0183331654) \\ \hline
    aTpsnriooitsinehpCXrXXXX &(14.4149989054) \\ \hline
    srnpTaistooihCpeniXXXXrX &(14.438332145399999)\\ \hline
    rsTnapsiotioChnpieXXrXXX &(13.4733328354) \\ \hline
    nrpsTatsoioipCehniXXXXrX &(13.5466661254) \\ \hline
    rTpasnsooiitCneihpXrXXXX &(13.2499996654) \\ \hline
    aTnspriotiosinpheCXrXXXX &(15.598331425400001) \\ \hline
    naspTrtiioospihenCXXXXrX &(14.964998475400002)\\ \hline
    pansrToitisoeiphCnXXXXXr &(14.9399984454) \\ \hline
\end{tabular}
\caption{Experimental results 1 of Fitness step}

\end{table}



\newpage
\subsection{experimentation of Selection step}
in experimentation of Selection steps, we will get the data are sorted depending on fitness values then we will select the best chromosomes:
\textsf{Experimental results 2 of Selection step:}\\
    \colorbox{blue!30}{\textsf{     No. of chromosomes is: 20}}\\
    \colorbox{blue!30}{\textsf{     length of chromosome is: 10}}\\
    \colorbox{blue!30}{\textsf{     CipherText: siaTrtposnernioXphiC}}\\
    \colorbox{blue!30}{\textsf{     PlainText: TranspositionCipher}}
\begin{table}[h!]
\centering
\begin{tabular}{{ l l l }}\hline
    (best Key(chromosome)) &(Plain text ) &(Fitness value )\\ \hline
    2  4  8  6  7  10    &ossipTrantheirpionCX&            17.2883305354\\ \hline 
    4  10  9  8  1  5    &rsostpnTaioiheXpCinr&            15.7466646554\\ \hline 
    9  5  2  10  7  4    &oastiprnsThniXrpoCei&            15.6816649754\\ \hline 
    6  4  9  10  3  2    &otrissnpaThXorieCpni&            15.6483316654\\ \hline 
    3  7  10  2  6  4    &nTstoripsaCieXhorpin&            15.5849980554\\ \hline 
    2  1  10  9  8  3    &istnopsrTareXChpioin&            15.5483314054\\ \hline 
    9  10  1  4  6  7    &aonTprtssinhCipoXier&            15.4316648154\\ \hline 
    10  3  7  6  9  4    &poitsTanrsphrXiinCoe&            15.2149983254\\ \hline 
    3  7  10  4  5  6    &nosTrtispaCheioXripn&            15.0066651054\\ \hline 
    4  2  5  3  1  10    &riTsaopnstorienhpCiX&            14.943331785400002\\ \hline 
\end{tabular}
\caption{Experimental results 2 Selection Step}
\end{table}


\textsf{Experimental results 2 of Selection step:}\\
    \colorbox{blue!30}{\textsf{     No. of chromosomes is: 20}}\\
    \colorbox{blue!30}{\textsf{     length of chromosome is: 10}}\\
    \colorbox{blue!30}{\textsf{     CipherText: siaTrtposnernioXphiC}}\\
    \colorbox{blue!30}{\textsf{     PlainText: TranspositionCipher}}
\begin{table}[h!]
\centering
\begin{tabular}{{ l l l }}\hline
    (best Key(chromosome)) &(Plain text ) &(Fitness value )\\ \hline
    2  4  8  6  7  10    &ossipTrantheirpionCX&            17.2883305354\\ \hline 
    4  10  9  8  1  5    &rsostpnTaioiheXpCinr&            15.7466646554\\ \hline 
    9  5  2  10  7  4    &oastiprnsThniXrpoCei&            15.6816649754\\ \hline 
    6  4  9  10  3  2    &otrissnpaThXorieCpni&            15.6483316654\\ \hline 
    3  7  10  2  6  4    &nTstoripsaCieXhorpin&            15.5849980554\\ \hline 
    2  1  10  9  8  3    &istnopsrTareXChpioin&            15.5483314054\\ \hline 
    9  10  1  4  6  7    &aonTprtssinhCipoXier&            15.4316648154\\ \hline 
    10  3  7  6  9  4    &poitsTanrsphrXiinCoe&            15.2149983254\\ \hline 
    3  7  10  4  5  6    &nosTrtispaCheioXripn&            15.0066651054\\ \hline 
    4  2  5  3  1  10    &riTsaopnstorienhpCiX&            14.943331785400002\\ \hline 
\end{tabular}
\caption{Experimental results 2 Selection Step}
\end{table}


\newpage
\subsection{experimentation of CrossOver step}
in this section, we will make a set of experiments to test every CrossOver operator, in the first part of the experiments will be applied, one crossover operator on the best parents to get half of the next population then mix those parents with their children to generate population completely, and in the second part of the experiments we will applied two Crossover operatores on the best parents  to generate population completely without keeping those parents to next generation, the experiments be:\\

\textsf{Experimental results 1 of CrossOver step:}\\
    \colorbox{blue!30}{\textsf{     No. of chromosomes is: 16}}\\
    \colorbox{blue!30}{\textsf{     CrossOver Operator: Multi Point}}\\
    \colorbox{blue!30}{\textsf{     Keeps best parent to next Population: true}}
    \begin{table}[H]
        \centering
        
        \begin{tabular}{{l l }}\hline
            Parent 1 &CJIBEVGYRTZWUMNSPFQOKHLDAX\\ \hline
            Parent 2 &CJYWORQPDMSNFEXGBTVIZLKUAH\\ \hline
            Parent 3 &GSCYDZXKNHTBMJEOPQLAIFWRUV\\ \hline
            Parent 4 &NTDLUYPVISGXBFMOEKZAHWJRQC\\ \hline
            Parent 5 &DMPBJXAGOSWEVUTNYHRCFZLKQI\\ \hline
            Parent 6   &VYDELMZQJTGSIRUNXAPFWKHOBC\\ \hline
            Parent 7 &FQRZKVHOCNEWGDPUMISLBTXYJA\\ \hline
            Parent 8 &RANHSXEPVUQDGTYBWKLZFJIOMC\\ \hline
          Child 1  &CJYWEVOQRTDSUMNGPFBIKHZLAX\\ \hline
          Child 2  &CJIVORGYDMWUFENSBTPQZLKXAH\\ \hline
          Child 3  &GSTLDZYXNHBOMJEKPQAWIFRCUV\\ \hline
          Child 4  &NTGDUYZXISMJBFOPEKLAHWRVQC\\ \hline
          Child 5   &DMELJXTGOSRNVUAPYHWKFZBCQI\\ \hline
          Child 6   &VYDPLMGOJTSEIRUNXAHFWKZQBC\\ \hline
          Child 7   &FQRHKVSXCNEPGDUYMIWLBTZOJA\\ \hline
          Child 8   &RAQZSXHOVUNEGTDPWKILFJBYMC\\ \hline
\end{tabular}

\caption{Experimental results 1 CrossOver Step}
\end{table}




\textsf{Experimental results 1 of CrossOver step:}\\
    \colorbox{blue!30}{\textsf{     No. of chromosomes is: 16}}\\
    \colorbox{blue!30}{\textsf{     CrossOver Operator: Multi Point}}\\
    \colorbox{blue!30}{\textsf{     Keeps best parent to next Population: true}}
    \begin{table}[H]
        \centering
        
        \begin{tabular}{{l l }}\hline
            Parent 1 &CJIBEVGYRTZWUMNSPFQOKHLDAX\\ \hline
            Parent 2 &CJYWORQPDMSNFEXGBTVIZLKUAH\\ \hline
            Parent 3 &GSCYDZXKNHTBMJEOPQLAIFWRUV\\ \hline
            Parent 4 &NTDLUYPVISGXBFMOEKZAHWJRQC\\ \hline
            Parent 5 &DMPBJXAGOSWEVUTNYHRCFZLKQI\\ \hline
            Parent 6   &VYDELMZQJTGSIRUNXAPFWKHOBC\\ \hline
            Parent 7 &FQRZKVHOCNEWGDPUMISLBTXYJA\\ \hline
            Parent 8 &RANHSXEPVUQDGTYBWKLZFJIOMC\\ \hline
          Child 1  &CJYWEVOQRTDSUMNGPFBIKHZLAX\\ \hline
          Child 2  &CJIVORGYDMWUFENSBTPQZLKXAH\\ \hline
          Child 3  &GSTLDZYXNHBOMJEKPQAWIFRCUV\\ \hline
          Child 4  &NTGDUYZXISMJBFOPEKLAHWRVQC\\ \hline
          Child 5   &DMELJXTGOSRNVUAPYHWKFZBCQI\\ \hline
          Child 6   &VYDPLMGOJTSEIRUNXAHFWKZQBC\\ \hline
          Child 7   &FQRHKVSXCNEPGDUYMIWLBTZOJA\\ \hline
          Child 8   &RAQZSXHOVUNEGTDPWKILFJBYMC\\ \hline
\end{tabular}

\caption{Experimental results 1 CrossOver Step}
\end{table}



\newpage
\subsection{experimentation of Mutation step}
At this stage, we will obtain the experimental results after the mutation process to obtain the final population that will continue to live to form other generations, the experimental results be:\\
\textsf{Experimental results 1 of Mutation step:}\\
    \colorbox{blue!30}{\textsf{     No. of chromosomes is: 16}}\\
    \colorbox{blue!30}{\textsf{     CrossOver Operator: Mutli Point}}\\
    \colorbox{blue!30}{\textsf{     Keeps best parent to next Population: true}}\\
    \colorbox{blue!30}{\textsf{     Two random Postions(2 bits of chromosome): 26 , 22}}\\
    \begin{table}[H]
        \centering
        \begin{tabular}{{ l l }}\hline
            \multicolumn{2}{c}{Final Population} \\ \hline
           
            chromosome 1 &CJIBEVGYRTZWUMNSPFQOKXLDAH\\ \hline
            chromosome 2 &CJYWORQPDMSNFEXGBTVIZHKUAL\\ \hline
            chromosome 3 &GSCYDZXKNHTBMJEOPQLAIVWRUF\\ \hline
            chromosome 4 & NTDLUYPVISGXBFMOEKZAHCJRQW\\ \hline
            chromosome 5 & DMPBJXAGOSWEVUTNYHRCFILKQZ\\ \hline
            chromosome 6 & VYDELMZQJTGSIRUNXAPFWCHOBK\\ \hline
            chromosome 7 & FQRZKVHOCNEWGDPUMISLBAXYJT\\ \hline
            chromosome 8 & RANHSXEPVUQDGTYBWKLZFCIOMJ\\ \hline
            chromosome 9 & CJYWEVOQRTDSUMNGPFBIKXZLAH\\ \hline
            chromosome 10 & CJIVORGYDMWUFENSBTPQZHKXAL\\ \hline
            chromosome 11 & GSTLDZYXNHBOMJEKPQAWIVRCUF\\ \hline
            chromosome 12 & NTGDUYZXISMJBFOPEKLAHCRVQW\\ \hline
            chromosome 13 & DMELJXTGOSRNVUAPYHWKFIBCQZ\\ \hline
            chromosome 14 & VYDPLMGOJTSEIRUNXAHFWCZQBK\\ \hline
            chromosome 15 &  FQRHKVSXCNEPGDUYMIWLBAZOJT\\ \hline
            chromosome 16 & RAQZSXHOVUNEGTDPWKILFCBYMJ\\ \hline
            
\end{tabular}
\caption{Experimental results 1 Mutation Step}
\end{table}



\textsf{Experimental results 1 of Mutation step:}\\
    \colorbox{blue!30}{\textsf{     No. of chromosomes is: 16}}\\
    \colorbox{blue!30}{\textsf{     CrossOver Operator: Mutli Point}}\\
    \colorbox{blue!30}{\textsf{     Keeps best parent to next Population: true}}\\
    \colorbox{blue!30}{\textsf{     Two random Postions(2 bits of chromosome): 26 , 22}}\\
    \begin{table}[H]
        \centering
        \begin{tabular}{{ l l }}\hline
            \multicolumn{2}{c}{Final Population} \\ \hline
           
            chromosome 1 &CJIBEVGYRTZWUMNSPFQOKXLDAH\\ \hline
            chromosome 2 &CJYWORQPDMSNFEXGBTVIZHKUAL\\ \hline
            chromosome 3 &GSCYDZXKNHTBMJEOPQLAIVWRUF\\ \hline
            chromosome 4 & NTDLUYPVISGXBFMOEKZAHCJRQW\\ \hline
            chromosome 5 & DMPBJXAGOSWEVUTNYHRCFILKQZ\\ \hline
            chromosome 6 & VYDELMZQJTGSIRUNXAPFWCHOBK\\ \hline
            chromosome 7 & FQRZKVHOCNEWGDPUMISLBAXYJT\\ \hline
            chromosome 8 & RANHSXEPVUQDGTYBWKLZFCIOMJ\\ \hline
            chromosome 9 & CJYWEVOQRTDSUMNGPFBIKXZLAH\\ \hline
            chromosome 10 & CJIVORGYDMWUFENSBTPQZHKXAL\\ \hline
            chromosome 11 & GSTLDZYXNHBOMJEKPQAWIVRCUF\\ \hline
            chromosome 12 & NTGDUYZXISMJBFOPEKLAHCRVQW\\ \hline
            chromosome 13 & DMELJXTGOSRNVUAPYHWKFIBCQZ\\ \hline
            chromosome 14 & VYDPLMGOJTSEIRUNXAHFWCZQBK\\ \hline
            chromosome 15 &  FQRHKVSXCNEPGDUYMIWLBAZOJT\\ \hline
            chromosome 16 & RAQZSXHOVUNEGTDPWKILFCBYMJ\\ \hline
            
\end{tabular}
\caption{Experimental results 1 Mutation Step}
\end{table}



\newpage


\section{experimentation of Substitution Cipher}
In this part of the work, we will make experiments for the working steps performed to each step of GAs of Substitution cipher.
\subsection{experimentation of population step}
As we said in the experimentation of population section of the Transposition Cipher that for each problem, there is different chromosome type then we used Integer chromosome Type, with substitution cipher problem the Integer chromosome Type will not work, so we have to use Character  chromosome Type, also in this problem the chromosome length will be constant which be 26 chars from A to Z are generated randomly and non-repetitive.
\textsf{Experimental results 2 of Population step:}\\
    \colorbox{blue!30}{\textsf{     No. of chromosomes is: 20}}\\
    \colorbox{blue!30}{\textsf{     length of chromosome is: 10}}\\

    \begin{table}[h!]
        \centering
        \begin{tabular}{{ l l }}\hline
            Key(chromosome 0)&9 10 1 4 6 7 5 2 8 3 \\ \hline
            Key(chromosome 1)&1 7 6 3 9 4 5 10 8 2 \\ \hline
            Key(chromosome 2)&6 4 9 10 3 2 8 1 5 7 \\ \hline
            Key(chromosome 3)&10 8 4 9 2 6 5 7 1 3 \\ \hline
            Key(chromosome 4)&7 3 1 9 8 5 2 4 10 6 \\ \hline
            Key(chromosome 5)&3 7 10 4 5 6 9 2 8 1 \\ \hline
            Key(chromosome 6)&2 8 1 3 6 7 9 10 4 5 \\ \hline
            Key(chromosome 7)&6 1 5 10 3 2 8 4 7 9 \\ \hline
            Key(chromosome 8)&2 4 8 6 7 10 5 1 3 9 \\ \hline
            Key(chromosome 9)&6 8 2 1 7 10 9 3 4 5 \\ \hline
            Key(chromosome 10)&10 7 8 3 9 1 4 5 6 2 \\ \hline
            Key(chromosome 11)&2 1 10 9 8 3 6 5 7 4 \\ \hline
            Key(chromosome 12)&10 3 7 6 9 4 1 2 5 8 \\ \hline
            Key(chromosome 13)&8 1 3 2 6 5 7 4 10 9 \\ \hline
            Key(chromosome 14)&9 8 6 1 10 4 2 3 7 5 \\ \hline
            Key(chromosome 15)&3 7 10 2 6 4 8 5 9 1 \\ \hline
            Key(chromosome 16)&9 5 2 10 7 4 6 1 3 8 \\ \hline
            Key(chromosome 17)&4 10 9 8 1 5 6 3 2 7 \\ \hline
            Key(chromosome 18)&4 2 5 3 1 10 7 6 9 8 \\ \hline
            Key(chromosome 19)&1 10 8 4 7 5 3 9 2 6 \\ \hline
\end{tabular}
\caption{Experimental results 2 of Population step}
\end{table}



\textsf{Experimental results 2 of Population step:}\\
    \colorbox{blue!30}{\textsf{     No. of chromosomes is: 20}}\\
    \colorbox{blue!30}{\textsf{     length of chromosome is: 10}}\\

    \begin{table}[h!]
        \centering
        \begin{tabular}{{ l l }}\hline
            Key(chromosome 0)&9 10 1 4 6 7 5 2 8 3 \\ \hline
            Key(chromosome 1)&1 7 6 3 9 4 5 10 8 2 \\ \hline
            Key(chromosome 2)&6 4 9 10 3 2 8 1 5 7 \\ \hline
            Key(chromosome 3)&10 8 4 9 2 6 5 7 1 3 \\ \hline
            Key(chromosome 4)&7 3 1 9 8 5 2 4 10 6 \\ \hline
            Key(chromosome 5)&3 7 10 4 5 6 9 2 8 1 \\ \hline
            Key(chromosome 6)&2 8 1 3 6 7 9 10 4 5 \\ \hline
            Key(chromosome 7)&6 1 5 10 3 2 8 4 7 9 \\ \hline
            Key(chromosome 8)&2 4 8 6 7 10 5 1 3 9 \\ \hline
            Key(chromosome 9)&6 8 2 1 7 10 9 3 4 5 \\ \hline
            Key(chromosome 10)&10 7 8 3 9 1 4 5 6 2 \\ \hline
            Key(chromosome 11)&2 1 10 9 8 3 6 5 7 4 \\ \hline
            Key(chromosome 12)&10 3 7 6 9 4 1 2 5 8 \\ \hline
            Key(chromosome 13)&8 1 3 2 6 5 7 4 10 9 \\ \hline
            Key(chromosome 14)&9 8 6 1 10 4 2 3 7 5 \\ \hline
            Key(chromosome 15)&3 7 10 2 6 4 8 5 9 1 \\ \hline
            Key(chromosome 16)&9 5 2 10 7 4 6 1 3 8 \\ \hline
            Key(chromosome 17)&4 10 9 8 1 5 6 3 2 7 \\ \hline
            Key(chromosome 18)&4 2 5 3 1 10 7 6 9 8 \\ \hline
            Key(chromosome 19)&1 10 8 4 7 5 3 9 2 6 \\ \hline
\end{tabular}
\caption{Experimental results 2 of Population step}
\end{table}



\newpage
\subsection{experimentation of Substitution Cipher step}
\begin{tcolorbox}[breakable,notitle,boxrule=0pt,colback=blue!20,colframe=blue!20]
    {
    \textsf{Experimental results 1 of Substitution Cipher step:}\\
    \textsf{     No. of chromosomes is: 16}\\
    \textsf{     CipherText: pnTsarotoiisepnhiCXXrXXX}\\
    \textsf{     PlainText: TranspositionCipher}
    }
    \end{tcolorbox}
\begin{table}[H]
\centering
\begin{tabular}{l l}\hline
    PlainTex0 &YBSQOQZGYVXOVBGUZIBD\\ \hline
PlainTex1 &NXVMKMLWNCBKCXWILJXQ\\ \hline
PlainTex2 &TYCXJXSITGBJGYIFSOYD\\ \hline
PlainTex3 &QRCOFOSNQMEFMRNDSYRI\\ \hline
PlainTex4 &XFZPLPVQXKGLKFQCVBFH\\ \hline
PlainTex5 &PJVFAFNUPRLARJUONQJD\\ \hline
PlainTex6 &LVRZKZBYLTSKTVYOBPVN\\ \hline
PlainTex7 &RODCHCNTRUJHUOTKNGOP\\ \hline
PlainTex8 &YWFNHNMKYGQHGWKCMDWE\\ \hline
PlainTex9 &VDCITIGXVEOTEDXUGPDY\\ \hline
PlainTex10 &LNGAQAOELJDQJNEROKNC\\ \hline
PlainTex11 &DRBYQYGHDKAQKRHOGZRL\\ \hline
PlainTex12 &NROCMCDUNGVMGRUSDZRY\\ \hline
PlainTex13 &ZINAKAOMZFUKFIMROVID\\ \hline
PlainTex14 &SCFLILUPSDKIDCPYUOCR\\ \hline
PlainTex15 &FAXLCLMKFPQCPAKNMDAU\\ \hline
\end{tabular}
\caption{Experimental results 1 Substitution Cipher}

\end{table}



\begin{tcolorbox}[breakable,notitle,boxrule=0pt,colback=blue!20,colframe=blue!20]
    {
    \textsf{Experimental results 1 of Substitution Cipher step:}\\
    \textsf{     No. of chromosomes is: 16}\\
    \textsf{     CipherText: pnTsarotoiisepnhiCXXrXXX}\\
    \textsf{     PlainText: TranspositionCipher}
    }
    \end{tcolorbox}
\begin{table}[H]
\centering
\begin{tabular}{l l}\hline
    PlainTex0 &YBSQOQZGYVXOVBGUZIBD\\ \hline
PlainTex1 &NXVMKMLWNCBKCXWILJXQ\\ \hline
PlainTex2 &TYCXJXSITGBJGYIFSOYD\\ \hline
PlainTex3 &QRCOFOSNQMEFMRNDSYRI\\ \hline
PlainTex4 &XFZPLPVQXKGLKFQCVBFH\\ \hline
PlainTex5 &PJVFAFNUPRLARJUONQJD\\ \hline
PlainTex6 &LVRZKZBYLTSKTVYOBPVN\\ \hline
PlainTex7 &RODCHCNTRUJHUOTKNGOP\\ \hline
PlainTex8 &YWFNHNMKYGQHGWKCMDWE\\ \hline
PlainTex9 &VDCITIGXVEOTEDXUGPDY\\ \hline
PlainTex10 &LNGAQAOELJDQJNEROKNC\\ \hline
PlainTex11 &DRBYQYGHDKAQKRHOGZRL\\ \hline
PlainTex12 &NROCMCDUNGVMGRUSDZRY\\ \hline
PlainTex13 &ZINAKAOMZFUKFIMROVID\\ \hline
PlainTex14 &SCFLILUPSDKIDCPYUOCR\\ \hline
PlainTex15 &FAXLCLMKFPQCPAKNMDAU\\ \hline
\end{tabular}
\caption{Experimental results 1 Substitution Cipher}

\end{table}



\newpage
\subsection{experimentation of Fitness step}
\\\textsf{Experimental results 1 of Fitness step:}\\
    \colorbox{blue!30}{\textsf{     No. of chromosomes is: 12}}\\
    \colorbox{blue!30}{\textsf{     length of chromosome is: 6}}\\
    \colorbox{blue!30}{\textsf{     CipherText: pnTsarotoiisepnhiCXXrXXX}}\\
    \colorbox{blue!30}{\textsf{     PlainText: TranspositionCipher}}

\begin{table}[h!]
\centering
\begin{tabular}{l l}
    \hline
    \cellcolor[gray]{0.9} PlainTexs& \cellcolor[gray]{0.9} Fitness\\ \hline
    TsrnapoistionhCpierXXXXX &(15.273331515399999)\\ \hline
    TspanroioitsnheipCrXXXXX &(14.738331975400001) \\ \hline
    pTrsnaoositienChpiXrXXXX &(13.836665915400001) \\ \hline
    psTnraoiotsiehnpCiXXrXXX &(13.0183331654) \\ \hline
    aTpsnriooitsinehpCXrXXXX &(14.4149989054) \\ \hline
    srnpTaistooihCpeniXXXXrX &(14.438332145399999)\\ \hline
    rsTnapsiotioChnpieXXrXXX &(13.4733328354) \\ \hline
    nrpsTatsoioipCehniXXXXrX &(13.5466661254) \\ \hline
    rTpasnsooiitCneihpXrXXXX &(13.2499996654) \\ \hline
    aTnspriotiosinpheCXrXXXX &(15.598331425400001) \\ \hline
    naspTrtiioospihenCXXXXrX &(14.964998475400002)\\ \hline
    pansrToitisoeiphCnXXXXXr &(14.9399984454) \\ \hline
\end{tabular}
\caption{Experimental results 1 of Fitness step}

\end{table}



\\\textsf{Experimental results 1 of Fitness step:}\\
    \colorbox{blue!30}{\textsf{     No. of chromosomes is: 12}}\\
    \colorbox{blue!30}{\textsf{     length of chromosome is: 6}}\\
    \colorbox{blue!30}{\textsf{     CipherText: pnTsarotoiisepnhiCXXrXXX}}\\
    \colorbox{blue!30}{\textsf{     PlainText: TranspositionCipher}}

\begin{table}[h!]
\centering
\begin{tabular}{l l}
    \hline
    \cellcolor[gray]{0.9} PlainTexs& \cellcolor[gray]{0.9} Fitness\\ \hline
    TsrnapoistionhCpierXXXXX &(15.273331515399999)\\ \hline
    TspanroioitsnheipCrXXXXX &(14.738331975400001) \\ \hline
    pTrsnaoositienChpiXrXXXX &(13.836665915400001) \\ \hline
    psTnraoiotsiehnpCiXXrXXX &(13.0183331654) \\ \hline
    aTpsnriooitsinehpCXrXXXX &(14.4149989054) \\ \hline
    srnpTaistooihCpeniXXXXrX &(14.438332145399999)\\ \hline
    rsTnapsiotioChnpieXXrXXX &(13.4733328354) \\ \hline
    nrpsTatsoioipCehniXXXXrX &(13.5466661254) \\ \hline
    rTpasnsooiitCneihpXrXXXX &(13.2499996654) \\ \hline
    aTnspriotiosinpheCXrXXXX &(15.598331425400001) \\ \hline
    naspTrtiioospihenCXXXXrX &(14.964998475400002)\\ \hline
    pansrToitisoeiphCnXXXXXr &(14.9399984454) \\ \hline
\end{tabular}
\caption{Experimental results 1 of Fitness step}

\end{table}



\newpage
\subsection{experimentation of Selection step}
\textsf{Experimental results 2 of Selection step:}\\
    \colorbox{blue!30}{\textsf{     No. of chromosomes is: 20}}\\
    \colorbox{blue!30}{\textsf{     length of chromosome is: 10}}\\
    \colorbox{blue!30}{\textsf{     CipherText: siaTrtposnernioXphiC}}\\
    \colorbox{blue!30}{\textsf{     PlainText: TranspositionCipher}}
\begin{table}[h!]
\centering
\begin{tabular}{{ l l l }}\hline
    (best Key(chromosome)) &(Plain text ) &(Fitness value )\\ \hline
    2  4  8  6  7  10    &ossipTrantheirpionCX&            17.2883305354\\ \hline 
    4  10  9  8  1  5    &rsostpnTaioiheXpCinr&            15.7466646554\\ \hline 
    9  5  2  10  7  4    &oastiprnsThniXrpoCei&            15.6816649754\\ \hline 
    6  4  9  10  3  2    &otrissnpaThXorieCpni&            15.6483316654\\ \hline 
    3  7  10  2  6  4    &nTstoripsaCieXhorpin&            15.5849980554\\ \hline 
    2  1  10  9  8  3    &istnopsrTareXChpioin&            15.5483314054\\ \hline 
    9  10  1  4  6  7    &aonTprtssinhCipoXier&            15.4316648154\\ \hline 
    10  3  7  6  9  4    &poitsTanrsphrXiinCoe&            15.2149983254\\ \hline 
    3  7  10  4  5  6    &nosTrtispaCheioXripn&            15.0066651054\\ \hline 
    4  2  5  3  1  10    &riTsaopnstorienhpCiX&            14.943331785400002\\ \hline 
\end{tabular}
\caption{Experimental results 2 Selection Step}
\end{table}


\textsf{Experimental results 2 of Selection step:}\\
    \colorbox{blue!30}{\textsf{     No. of chromosomes is: 20}}\\
    \colorbox{blue!30}{\textsf{     length of chromosome is: 10}}\\
    \colorbox{blue!30}{\textsf{     CipherText: siaTrtposnernioXphiC}}\\
    \colorbox{blue!30}{\textsf{     PlainText: TranspositionCipher}}
\begin{table}[h!]
\centering
\begin{tabular}{{ l l l }}\hline
    (best Key(chromosome)) &(Plain text ) &(Fitness value )\\ \hline
    2  4  8  6  7  10    &ossipTrantheirpionCX&            17.2883305354\\ \hline 
    4  10  9  8  1  5    &rsostpnTaioiheXpCinr&            15.7466646554\\ \hline 
    9  5  2  10  7  4    &oastiprnsThniXrpoCei&            15.6816649754\\ \hline 
    6  4  9  10  3  2    &otrissnpaThXorieCpni&            15.6483316654\\ \hline 
    3  7  10  2  6  4    &nTstoripsaCieXhorpin&            15.5849980554\\ \hline 
    2  1  10  9  8  3    &istnopsrTareXChpioin&            15.5483314054\\ \hline 
    9  10  1  4  6  7    &aonTprtssinhCipoXier&            15.4316648154\\ \hline 
    10  3  7  6  9  4    &poitsTanrsphrXiinCoe&            15.2149983254\\ \hline 
    3  7  10  4  5  6    &nosTrtispaCheioXripn&            15.0066651054\\ \hline 
    4  2  5  3  1  10    &riTsaopnstorienhpCiX&            14.943331785400002\\ \hline 
\end{tabular}
\caption{Experimental results 2 Selection Step}
\end{table}


\newpage
\subsection{experimentation of CrossOver step}

\textsf{Experimental results 1 of CrossOver step:}\\
    \colorbox{blue!30}{\textsf{     No. of chromosomes is: 16}}\\
    \colorbox{blue!30}{\textsf{     CrossOver Operator: Multi Point}}\\
    \colorbox{blue!30}{\textsf{     Keeps best parent to next Population: true}}
    \begin{table}[H]
        \centering
        
        \begin{tabular}{{l l }}\hline
            Parent 1 &CJIBEVGYRTZWUMNSPFQOKHLDAX\\ \hline
            Parent 2 &CJYWORQPDMSNFEXGBTVIZLKUAH\\ \hline
            Parent 3 &GSCYDZXKNHTBMJEOPQLAIFWRUV\\ \hline
            Parent 4 &NTDLUYPVISGXBFMOEKZAHWJRQC\\ \hline
            Parent 5 &DMPBJXAGOSWEVUTNYHRCFZLKQI\\ \hline
            Parent 6   &VYDELMZQJTGSIRUNXAPFWKHOBC\\ \hline
            Parent 7 &FQRZKVHOCNEWGDPUMISLBTXYJA\\ \hline
            Parent 8 &RANHSXEPVUQDGTYBWKLZFJIOMC\\ \hline
          Child 1  &CJYWEVOQRTDSUMNGPFBIKHZLAX\\ \hline
          Child 2  &CJIVORGYDMWUFENSBTPQZLKXAH\\ \hline
          Child 3  &GSTLDZYXNHBOMJEKPQAWIFRCUV\\ \hline
          Child 4  &NTGDUYZXISMJBFOPEKLAHWRVQC\\ \hline
          Child 5   &DMELJXTGOSRNVUAPYHWKFZBCQI\\ \hline
          Child 6   &VYDPLMGOJTSEIRUNXAHFWKZQBC\\ \hline
          Child 7   &FQRHKVSXCNEPGDUYMIWLBTZOJA\\ \hline
          Child 8   &RAQZSXHOVUNEGTDPWKILFJBYMC\\ \hline
\end{tabular}

\caption{Experimental results 1 CrossOver Step}
\end{table}




\textsf{Experimental results 1 of CrossOver step:}\\
    \colorbox{blue!30}{\textsf{     No. of chromosomes is: 16}}\\
    \colorbox{blue!30}{\textsf{     CrossOver Operator: Multi Point}}\\
    \colorbox{blue!30}{\textsf{     Keeps best parent to next Population: true}}
    \begin{table}[H]
        \centering
        
        \begin{tabular}{{l l }}\hline
            Parent 1 &CJIBEVGYRTZWUMNSPFQOKHLDAX\\ \hline
            Parent 2 &CJYWORQPDMSNFEXGBTVIZLKUAH\\ \hline
            Parent 3 &GSCYDZXKNHTBMJEOPQLAIFWRUV\\ \hline
            Parent 4 &NTDLUYPVISGXBFMOEKZAHWJRQC\\ \hline
            Parent 5 &DMPBJXAGOSWEVUTNYHRCFZLKQI\\ \hline
            Parent 6   &VYDELMZQJTGSIRUNXAPFWKHOBC\\ \hline
            Parent 7 &FQRZKVHOCNEWGDPUMISLBTXYJA\\ \hline
            Parent 8 &RANHSXEPVUQDGTYBWKLZFJIOMC\\ \hline
          Child 1  &CJYWEVOQRTDSUMNGPFBIKHZLAX\\ \hline
          Child 2  &CJIVORGYDMWUFENSBTPQZLKXAH\\ \hline
          Child 3  &GSTLDZYXNHBOMJEKPQAWIFRCUV\\ \hline
          Child 4  &NTGDUYZXISMJBFOPEKLAHWRVQC\\ \hline
          Child 5   &DMELJXTGOSRNVUAPYHWKFZBCQI\\ \hline
          Child 6   &VYDPLMGOJTSEIRUNXAHFWKZQBC\\ \hline
          Child 7   &FQRHKVSXCNEPGDUYMIWLBTZOJA\\ \hline
          Child 8   &RAQZSXHOVUNEGTDPWKILFJBYMC\\ \hline
\end{tabular}

\caption{Experimental results 1 CrossOver Step}
\end{table}



\newpage
\subsection{experimentation of Mutation step}
\textsf{Experimental results 1 of Mutation step:}\\
    \colorbox{blue!30}{\textsf{     No. of chromosomes is: 16}}\\
    \colorbox{blue!30}{\textsf{     CrossOver Operator: Mutli Point}}\\
    \colorbox{blue!30}{\textsf{     Keeps best parent to next Population: true}}\\
    \colorbox{blue!30}{\textsf{     Two random Postions(2 bits of chromosome): 26 , 22}}\\
    \begin{table}[H]
        \centering
        \begin{tabular}{{ l l }}\hline
            \multicolumn{2}{c}{Final Population} \\ \hline
           
            chromosome 1 &CJIBEVGYRTZWUMNSPFQOKXLDAH\\ \hline
            chromosome 2 &CJYWORQPDMSNFEXGBTVIZHKUAL\\ \hline
            chromosome 3 &GSCYDZXKNHTBMJEOPQLAIVWRUF\\ \hline
            chromosome 4 & NTDLUYPVISGXBFMOEKZAHCJRQW\\ \hline
            chromosome 5 & DMPBJXAGOSWEVUTNYHRCFILKQZ\\ \hline
            chromosome 6 & VYDELMZQJTGSIRUNXAPFWCHOBK\\ \hline
            chromosome 7 & FQRZKVHOCNEWGDPUMISLBAXYJT\\ \hline
            chromosome 8 & RANHSXEPVUQDGTYBWKLZFCIOMJ\\ \hline
            chromosome 9 & CJYWEVOQRTDSUMNGPFBIKXZLAH\\ \hline
            chromosome 10 & CJIVORGYDMWUFENSBTPQZHKXAL\\ \hline
            chromosome 11 & GSTLDZYXNHBOMJEKPQAWIVRCUF\\ \hline
            chromosome 12 & NTGDUYZXISMJBFOPEKLAHCRVQW\\ \hline
            chromosome 13 & DMELJXTGOSRNVUAPYHWKFIBCQZ\\ \hline
            chromosome 14 & VYDPLMGOJTSEIRUNXAHFWCZQBK\\ \hline
            chromosome 15 &  FQRHKVSXCNEPGDUYMIWLBAZOJT\\ \hline
            chromosome 16 & RAQZSXHOVUNEGTDPWKILFCBYMJ\\ \hline
            
\end{tabular}
\caption{Experimental results 1 Mutation Step}
\end{table}



\textsf{Experimental results 1 of Mutation step:}\\
    \colorbox{blue!30}{\textsf{     No. of chromosomes is: 16}}\\
    \colorbox{blue!30}{\textsf{     CrossOver Operator: Mutli Point}}\\
    \colorbox{blue!30}{\textsf{     Keeps best parent to next Population: true}}\\
    \colorbox{blue!30}{\textsf{     Two random Postions(2 bits of chromosome): 26 , 22}}\\
    \begin{table}[H]
        \centering
        \begin{tabular}{{ l l }}\hline
            \multicolumn{2}{c}{Final Population} \\ \hline
           
            chromosome 1 &CJIBEVGYRTZWUMNSPFQOKXLDAH\\ \hline
            chromosome 2 &CJYWORQPDMSNFEXGBTVIZHKUAL\\ \hline
            chromosome 3 &GSCYDZXKNHTBMJEOPQLAIVWRUF\\ \hline
            chromosome 4 & NTDLUYPVISGXBFMOEKZAHCJRQW\\ \hline
            chromosome 5 & DMPBJXAGOSWEVUTNYHRCFILKQZ\\ \hline
            chromosome 6 & VYDELMZQJTGSIRUNXAPFWCHOBK\\ \hline
            chromosome 7 & FQRZKVHOCNEWGDPUMISLBAXYJT\\ \hline
            chromosome 8 & RANHSXEPVUQDGTYBWKLZFCIOMJ\\ \hline
            chromosome 9 & CJYWEVOQRTDSUMNGPFBIKXZLAH\\ \hline
            chromosome 10 & CJIVORGYDMWUFENSBTPQZHKXAL\\ \hline
            chromosome 11 & GSTLDZYXNHBOMJEKPQAWIVRCUF\\ \hline
            chromosome 12 & NTGDUYZXISMJBFOPEKLAHCRVQW\\ \hline
            chromosome 13 & DMELJXTGOSRNVUAPYHWKFIBCQZ\\ \hline
            chromosome 14 & VYDPLMGOJTSEIRUNXAHFWCZQBK\\ \hline
            chromosome 15 &  FQRHKVSXCNEPGDUYMIWLBAZOJT\\ \hline
            chromosome 16 & RAQZSXHOVUNEGTDPWKILFCBYMJ\\ \hline
            
\end{tabular}
\caption{Experimental results 1 Mutation Step}
\end{table}



\newpage