\chapter{Experimentation}
In this chapter, we will write all the issues related to the experimentation phase of the study, from the first decision-making to the last decision-making to validate the solutions. Throughout the following paragraphs, Arguments will be presented in defense of the decisions to be made and a sufficiently detailed description of the experiment will be presented so that the results can be easily interpreted.in this chapter we will get experimental results for each step of the application from population steps to mutation steps with giving the details of each step And make comparisons of the experimental results, These experiments are divided into two parts that depend on the cipher type.
so, we have two stages of the experimentation, which are:

\section{experimentation of Transposition Cipher}
In this part of the work, we will make experiments for the working steps performed to each step of GAs of Transposition cipher.
\subsection{experimentation of population step}
In this part of the work, we will make experiments for the working steps performed to each step of GAs of Transposition cipher.
in this section, we will make an experiment to generate a set of chromosomes(individuals) which represent the first population,  in this case, each chromosome has a set of numbers which represent DNA Which can not be repeated in another chromosome, for each problem, there is different chromosome type that we can define the chromosome (also sometimes called a genotype) is a set of parameters which define a proposed solution to the problem that the genetic algorithm is trying to solve. The set of all solutions is known as the population. The chromosome is often represented as a binary string, although a wide variety of other data structures are also used \cite{IVGeneticAlgorithm}.\\
\textsf{Experimental results 1 of Population step:}\\
    \colorbox{blue!30}{\textsf{     No. of chromosomes is: 12}}\\
    \colorbox{blue!30}{\textsf{     length of chromosome is: 6}}

\begin{table}[h!]
\centering
\begin{tabular}{l l}\hline
    Key(chromosome 0)&6 4 1 2 5 3\\ \hline 
    Key(chromosome 1)&3 5 1 2 4 6 \\ \hline 
    Key(chromosome 2)&1 5 2 4 6 3 \\ \hline 
    Key(chromosome 3)&1 4 3 2 6 5 \\ \hline 
    Key(chromosome 4)&3 5 2 4 1 6 \\ \hline 
    Key(chromosome 5)&4 3 5 1 6 2 \\ \hline 
    Key(chromosome 6)&6 4 3 2 5 1 \\ \hline 
    Key(chromosome 7)&3 1 5 4 6 2 \\ \hline 
    Key(chromosome 8)&3 6 2 5 4 1 \\ \hline 
    Key(chromosome 9)&5 3 2 4 1 6 \\ \hline 
    Key(chromosome 10)&4 1 5 3 2 6 \\ \hline 
    Key(chromosome 11)&1 3 6 4 2 5 \\ \hline  
\end{tabular}
\caption{Experimental results 1 Population step}

\end{table}



\textsf{Experimental results 1 of Population step:}\\
    \colorbox{blue!30}{\textsf{     No. of chromosomes is: 16}}\\

\begin{table}[H]
\centering
\begin{tabular}{l l}\hline
    Key(Chromosome0)&SEDJXKPIBCTRLVGZAOYQFHWUMN\\ \hline
    Key(Chromosome1)&VRQDBOFJXHYASCWLGKNMPZUITE\\ \hline
    Key(Chromosome2)&CMDABENOYRLKQGISPJTXVUWFZH\\ \hline
    Key(Chromosome3)&CJIBEVGYRTZWUMNSPFQOKHLDAX\\ \hline
    Key(Chromosome4)&ZDHTGNYBFMJRUKQVILXPWAECOS\\ \hline
    Key(Chromosome5)&VYDELMZQJTGSIRUNXAPFWKHOBC\\ \hline
    Key(Chromosome6)&RANHSXEPVUQDGTYBWKLZFJIOMC\\ \hline
    Key(Chromosome7)&DMPBJXAGOSWEVUTNYHRCFZLKQI\\ \hline
    Key(Chromosome8)&FZEBQTVDWXAIOGKMSHYNUPLCJR\\ \hline
    Key(Chromosome9)&CJYWORQPDMSNFEXGBTVIZLKUAH\\ \hline
    Key(Chromosome10)&GSCYDZXKNHTBMJEOPQLAIFWRUV\\ \hline
    Key(Chromosome11)&BNLVATPZRUSMCKHGXQDYIEJOWF\\ \hline
    Key(Chromosome12)&OWYFVAEZRJPHKGUDTMNCQIXSLB\\ \hline
    Key(Chromosome13)&NTDLUYPVISGXBFMOEKZAHWJRQC\\ \hline
    Key(Chromosome14)&FQRZKVHOCNEWGDPUMISLBTXYJA\\ \hline
    Key(Chromosome15)&XSUIQETDAYJRBPKMOCFLGVHNWZ\\ \hline
\end{tabular}
\caption{Experimental results 1 Population step}

\end{table}



\newpage
\subsection{experimentation of Transposition Cipher step}
in this section, we will try to break the ciphertext by using the decoding Transposition Algorithm to get the experimental results which will be:
\textsf{Experimental results 2 of Transposition Cipher step:}\\
    \colorbox{blue!30}{\textsf{     No. of chromosomes is: 20}}\\
    \colorbox{blue!30}{\textsf{     length of chromosome is: 10}}\\
    \colorbox{blue!30}{\textsf{     CipherText: siaTrtposnernioXphiC}}\\
    \colorbox{blue!30}{\textsf{     PlainText: TranspositionCipher}}
\begin{table}[h!]
\centering
\begin{tabular}{l l}\hline
    Plaintexts 0&aonTprtssinhCipoXier\\ \hline
    Plaintexts 1&snTtpaisroeCiXpnrioh\\ \hline
    Plaintexts 2&otrissnpaThXorieCpni\\ \hline
    Plaintexts 3&srnaptoiTsioCnpXhrie\\ \hline
    Plaintexts 4&apiotnsrTsnprhXCeoii\\ \hline
    Plaintexts 5&nosTrtispaCheioXripn\\ \hline
    Plaintexts 6&asTsnrtiponeiiCoXrph\\ \hline
    Plaintexts 7&itroasspnTrXohneipCi\\ \hline
    Plaintexts 8&ossipTrantheirpionCX\\ \hline
    Plaintexts 9&TaosnsriptinhiCeorpX\\ \hline
    Plaintexts 10&tnTposiarsXCiphirnoe\\ \hline
    Plaintexts 11&istnopsrTareXChpioin\\ \hline
    Plaintexts 12&poitsTanrsphrXiinCoe\\ \hline
    Plaintexts 13&iTaotrpsnsrinhXopeCi\\ \hline
    Plaintexts 14&TpotnasisriphXCnireo\\ \hline
    Plaintexts 15&nTstoripsaCieXhorpin\\ \hline
    Plaintexts 16&oastiprnsThniXrpoCei\\ \hline
    Plaintexts 17&rsostpnTaioiheXpCinr\\ \hline
    Plaintexts 18&riTsaopnstorienhpCiX\\ \hline
    Plaintexts 19&sspTtnraoieipiXConhr\\ \hline
\end{tabular}
\caption{Experimental results 2 Transposition Cipher}

\end{table}
\textsf{Experimental results 2 of Transposition Cipher step:}\\
    \colorbox{blue!30}{\textsf{     No. of chromosomes is: 20}}\\
    \colorbox{blue!30}{\textsf{     length of chromosome is: 10}}\\
    \colorbox{blue!30}{\textsf{     CipherText: siaTrtposnernioXphiC}}\\
    \colorbox{blue!30}{\textsf{     PlainText: TranspositionCipher}}
\begin{table}[h!]
\centering
\begin{tabular}{l l}\hline
    Plaintexts 0&aonTprtssinhCipoXier\\ \hline
    Plaintexts 1&snTtpaisroeCiXpnrioh\\ \hline
    Plaintexts 2&otrissnpaThXorieCpni\\ \hline
    Plaintexts 3&srnaptoiTsioCnpXhrie\\ \hline
    Plaintexts 4&apiotnsrTsnprhXCeoii\\ \hline
    Plaintexts 5&nosTrtispaCheioXripn\\ \hline
    Plaintexts 6&asTsnrtiponeiiCoXrph\\ \hline
    Plaintexts 7&itroasspnTrXohneipCi\\ \hline
    Plaintexts 8&ossipTrantheirpionCX\\ \hline
    Plaintexts 9&TaosnsriptinhiCeorpX\\ \hline
    Plaintexts 10&tnTposiarsXCiphirnoe\\ \hline
    Plaintexts 11&istnopsrTareXChpioin\\ \hline
    Plaintexts 12&poitsTanrsphrXiinCoe\\ \hline
    Plaintexts 13&iTaotrpsnsrinhXopeCi\\ \hline
    Plaintexts 14&TpotnasisriphXCnireo\\ \hline
    Plaintexts 15&nTstoripsaCieXhorpin\\ \hline
    Plaintexts 16&oastiprnsThniXrpoCei\\ \hline
    Plaintexts 17&rsostpnTaioiheXpCinr\\ \hline
    Plaintexts 18&riTsaopnstorienhpCiX\\ \hline
    Plaintexts 19&sspTtnraoieipiXConhr\\ \hline
\end{tabular}
\caption{Experimental results 2 Transposition Cipher}

\end{table}
\newpage
\subsection{experimentation of Fitness step}
now, we will get the preliminary experimental results to fitness step which will be:\\
\textsf{Experimental results 2 of Fitness step:}\\
    \colorbox{blue!30}{\textsf{     No. of chromosomes is: 20}}\\
    \colorbox{blue!30}{\textsf{     length of chromosome is: 10}}\\
    \colorbox{blue!30}{\textsf{     CipherText: siaTrtposnernioXphiC}}\\
    \colorbox{blue!30}{\textsf{     PlainText: TranspositionCipher}}

\begin{table}[H]
\centering
\begin{tabular}{l l}
    \hline
    \cellcolor[gray]{0.9} PlainTexs& \cellcolor[gray]{0.9} Fitness\\ \hline
    aonTprtssinhCipoXier&(15.4316648154)\\ \hline  
snTtpaisroeCiXpnrioh&(13.3783329054)\\ \hline  
otrissnpaThXorieCpni&(15.6483316654)\\ \hline  
srnaptoiTsioCnpXhrie&(13.3466662654)\\ \hline  
apiotnsrTsnprhXCeoii&(12.4366669154)\\ \hline  
nosTrtispaCheioXripn&(15.0066651054)\\ \hline  
asTsnrtiponeiiCoXrph&(14.4783321054)\\ \hline  
itroasspnTrXohneipCi&(13.799999275400001)\\ \hline  
ossipTrantheirpionCX&(17.2883305354)\\ \hline  
TaosnsriptinhiCeorpX&(14.4116655554)\\ \hline  
tnTposiarsXCiphirnoe&(13.4383328654)\\ \hline  
istnopsrTareXChpioin&(15.5483314054)\\ \hline  
poitsTanrsphrXiinCoe&(15.2149983254)\\ \hline  
iTaotrpsnsrinhXopeCi&(13.8866659454)\\ \hline  
TpotnasisriphXCnireo&(13.8783325354)\\ \hline  
nTstoripsaCieXhorpin&(15.5849980554)\\ \hline  
oastiprnsThniXrpoCei&(15.6816649754)\\ \hline  
rsostpnTaioiheXpCinr&(15.7466646554)\\ \hline  
riTsaopnstorienhpCiX&(14.943331785400002)\\ \hline  
sspTtnraoieipiXConhr&(13.0416664754)\\ \hline  
\end{tabular}
\caption{Experimental results 2 of Fitness step}

\end{table}



\textsf{Experimental results 2 of Fitness step:}\\
    \colorbox{blue!30}{\textsf{     No. of chromosomes is: 20}}\\
    \colorbox{blue!30}{\textsf{     length of chromosome is: 10}}\\
    \colorbox{blue!30}{\textsf{     CipherText: siaTrtposnernioXphiC}}\\
    \colorbox{blue!30}{\textsf{     PlainText: TranspositionCipher}}

\begin{table}[H]
\centering
\begin{tabular}{l l}
    \hline
    \cellcolor[gray]{0.9} PlainTexs& \cellcolor[gray]{0.9} Fitness\\ \hline
    aonTprtssinhCipoXier&(15.4316648154)\\ \hline  
snTtpaisroeCiXpnrioh&(13.3783329054)\\ \hline  
otrissnpaThXorieCpni&(15.6483316654)\\ \hline  
srnaptoiTsioCnpXhrie&(13.3466662654)\\ \hline  
apiotnsrTsnprhXCeoii&(12.4366669154)\\ \hline  
nosTrtispaCheioXripn&(15.0066651054)\\ \hline  
asTsnrtiponeiiCoXrph&(14.4783321054)\\ \hline  
itroasspnTrXohneipCi&(13.799999275400001)\\ \hline  
ossipTrantheirpionCX&(17.2883305354)\\ \hline  
TaosnsriptinhiCeorpX&(14.4116655554)\\ \hline  
tnTposiarsXCiphirnoe&(13.4383328654)\\ \hline  
istnopsrTareXChpioin&(15.5483314054)\\ \hline  
poitsTanrsphrXiinCoe&(15.2149983254)\\ \hline  
iTaotrpsnsrinhXopeCi&(13.8866659454)\\ \hline  
TpotnasisriphXCnireo&(13.8783325354)\\ \hline  
nTstoripsaCieXhorpin&(15.5849980554)\\ \hline  
oastiprnsThniXrpoCei&(15.6816649754)\\ \hline  
rsostpnTaioiheXpCinr&(15.7466646554)\\ \hline  
riTsaopnstorienhpCiX&(14.943331785400002)\\ \hline  
sspTtnraoieipiXConhr&(13.0416664754)\\ \hline  
\end{tabular}
\caption{Experimental results 2 of Fitness step}

\end{table}



\newpage
\subsection{experimentation of Selection step}
in experimentation of Selection steps, we will get the data are sorted depending on fitness values then we will select the best chromosomes:
\textsf{Experimental results 2 of Selection step:}\\
    \colorbox{blue!30}{\textsf{     No. of chromosomes is: 24}}\\
    \colorbox{blue!30}{\textsf{     CipherText: pnTsarotoiisepnhiCXXrXXX}}\\
    \colorbox{blue!30}{\textsf{     PlainText: TranspositionCipher}}
\begin{table}[H]
\centering
\begin{tabular}{{ m{7cm} m{5.5cm} m{1cm}}}\hline
    (best Key(chromosome)) &(Plain text ) &(Fitness value )\\ \hline
 TQXUIEJONBYSZKPHCRAWDVGLFM    &ANTWRWHPAKIRKNPLHONX&            11.874\\ \hline 
 MZBWGVILEQJAKSNRCHUOFPYXTD    &UEMOHORNUSGHSENXRLEB&            10.976\\ \hline 
 SLNGHABEDCOVQPMRXYWKFIUTJZ    &WDSKYKRMWPHYPDMTREDN&            10.943\\ \hline 
 RWHLNQMEUDVZSOGBYKXPFTJACI    &XURPKPBGXONKOUGABEUH&            10.283\\ \hline 
 UBRKAHQGOZPMNCTJWLFDSVIYEX    &FOUDLDJTFCALCOTYJGOR&            9.808\\ \hline 
 ELGKODWTBANXPCRHVZJUQYFSIM    &JBEUZUHRJCOZCBRSHTBG&            9.335\\ \hline 
 JARUCOYVHBDTXISEFGZLKQWMNP    &ZHJLGLESZICGIHSMEVHR&            9.021\\ \hline 
 ZMRQGOYLFBIJPEXDHVSWACKNTU    &SFZWVWDXSEGVEFXNDLFR&            8.763\\ \hline 
 NPAMLEXYHOQFVIZKBCWUSDTRGJ    &WHNUCUKZWILCIHZRKYHA&            8.524\\ \hline 
 YCJWOQERHKXMPFVLINDSTUGZAB    &DHYSNSLVDFONFHVZLRHJ&            8.213\\ \hline 
 FSYPXBTCJZGHOVNIAWEQLDUMRK    &EJFQWQINEVXWVJNMICJY&            7.634\\ \hline 
 GTSLQFDOCAXZKEHYIBVPMRWUNJ    &VCGPBPYHVEQBECHUYOCS&            7.560\\ \hline 
 
\end{tabular}
\caption{Experimental results 2 Selection Step}
\end{table}



\textsf{Experimental results 2 of Selection step:}\\
    \colorbox{blue!30}{\textsf{     No. of chromosomes is: 24}}\\
    \colorbox{blue!30}{\textsf{     CipherText: pnTsarotoiisepnhiCXXrXXX}}\\
    \colorbox{blue!30}{\textsf{     PlainText: TranspositionCipher}}
\begin{table}[H]
\centering
\begin{tabular}{{ m{7cm} m{5.5cm} m{1cm}}}\hline
    (best Key(chromosome)) &(Plain text ) &(Fitness value )\\ \hline
 TQXUIEJONBYSZKPHCRAWDVGLFM    &ANTWRWHPAKIRKNPLHONX&            11.874\\ \hline 
 MZBWGVILEQJAKSNRCHUOFPYXTD    &UEMOHORNUSGHSENXRLEB&            10.976\\ \hline 
 SLNGHABEDCOVQPMRXYWKFIUTJZ    &WDSKYKRMWPHYPDMTREDN&            10.943\\ \hline 
 RWHLNQMEUDVZSOGBYKXPFTJACI    &XURPKPBGXONKOUGABEUH&            10.283\\ \hline 
 UBRKAHQGOZPMNCTJWLFDSVIYEX    &FOUDLDJTFCALCOTYJGOR&            9.808\\ \hline 
 ELGKODWTBANXPCRHVZJUQYFSIM    &JBEUZUHRJCOZCBRSHTBG&            9.335\\ \hline 
 JARUCOYVHBDTXISEFGZLKQWMNP    &ZHJLGLESZICGIHSMEVHR&            9.021\\ \hline 
 ZMRQGOYLFBIJPEXDHVSWACKNTU    &SFZWVWDXSEGVEFXNDLFR&            8.763\\ \hline 
 NPAMLEXYHOQFVIZKBCWUSDTRGJ    &WHNUCUKZWILCIHZRKYHA&            8.524\\ \hline 
 YCJWOQERHKXMPFVLINDSTUGZAB    &DHYSNSLVDFONFHVZLRHJ&            8.213\\ \hline 
 FSYPXBTCJZGHOVNIAWEQLDUMRK    &EJFQWQINEVXWVJNMICJY&            7.634\\ \hline 
 GTSLQFDOCAXZKEHYIBVPMRWUNJ    &VCGPBPYHVEQBECHUYOCS&            7.560\\ \hline 
 
\end{tabular}
\caption{Experimental results 2 Selection Step}
\end{table}



\newpage
\subsection{experimentation of CrossOver step}
in this section, we will make a set of experiments to test every CrossOver operator, in the first part of the experiments will be applied, one crossover operator on the best parents to get half of the next population then mix those parents with their children to generate population completely, and in the second part of the experiments we will applied two Crossover operatores on the best parents  to generate population completely without keeping those parents to next generation, the experiments be:\\
\\\textsf{Experimental results 1 of CrossOver step:}\\
    \colorbox{blue!30}{\textsf{     No. of chromosomes is: 12}}\\
    \colorbox{blue!30}{\textsf{     length of chromosome is: 6}}\\
    \colorbox{blue!30}{\textsf{     CrossOver Operator: Two Point}}\\
    \colorbox{blue!30}{\textsf{     Keeps best parent to next Population: true}}

    \begin{table}[H]
        \centering
        \begin{tabular}{{ l l }}\hline
            Parent 1& 5  3  2  4  1  6 \\ \hline
            Parent 2&  6  4  1  2  5  3 \\ \hline
            Parent 3&  4  1  5  3  2  6 \\ \hline
          Parent 4&  1  3  6  4  2  5 \\ \hline
          Parent 5&  3  5  1  2  4  6 \\ \hline
          Parent 6&  4  3  5  1  6  2 \\ \hline
          Child 1&  6  1  2  4  5  3 \\ \hline
          Child 2&  5  3  1  2  4  6 \\ \hline
          Child 3&   1  6  5  3  4  2 \\ \hline
          Child 4&   1  5  6  4  3  2 \\ \hline
          Child 5&  4  3  1  2  5  6 \\ \hline
          Child 6&   3  2  5  1  4  6\\ \hline
\end{tabular}
\caption{Experimental results 1 CrossOver Step}
\end{table}



\\\textsf{Experimental results 1 of CrossOver step:}\\
    \colorbox{blue!30}{\textsf{     No. of chromosomes is: 12}}\\
    \colorbox{blue!30}{\textsf{     length of chromosome is: 6}}\\
    \colorbox{blue!30}{\textsf{     CrossOver Operator: Two Point}}\\
    \colorbox{blue!30}{\textsf{     Keeps best parent to next Population: true}}

    \begin{table}[H]
        \centering
        \begin{tabular}{{ l l }}\hline
            Parent 1& 5  3  2  4  1  6 \\ \hline
            Parent 2&  6  4  1  2  5  3 \\ \hline
            Parent 3&  4  1  5  3  2  6 \\ \hline
          Parent 4&  1  3  6  4  2  5 \\ \hline
          Parent 5&  3  5  1  2  4  6 \\ \hline
          Parent 6&  4  3  5  1  6  2 \\ \hline
          Child 1&  6  1  2  4  5  3 \\ \hline
          Child 2&  5  3  1  2  4  6 \\ \hline
          Child 3&   1  6  5  3  4  2 \\ \hline
          Child 4&   1  5  6  4  3  2 \\ \hline
          Child 5&  4  3  1  2  5  6 \\ \hline
          Child 6&   3  2  5  1  4  6\\ \hline
\end{tabular}
\caption{Experimental results 1 CrossOver Step}
\end{table}



\newpage
\subsection{experimentation of Mutation step}
At this stage, we will obtain the experimental results after the mutation process to obtain the final population that will continue to live to form other generations, the experimental results be:\\
\textsf{Experimental results 2 of Mutation step:}\\
    \colorbox{blue!30}{\textsf{     No. of chromosomes is: 20}}\\
    \colorbox{blue!30}{\textsf{     length of chromosome is: 10}}\\
    \colorbox{blue!30}{\textsf{     CrossOver Operator:One Point and Crossing}}\\
    \colorbox{blue!30}{\textsf{     Keeps best parent to next Population: false}}\\
    \colorbox{blue!30}{\textsf{     Two random Postions(2 bits of chromosome): 3 , 5}}\\
    \begin{table}[H]
        \centering
        \begin{tabular}{{ l l }}\hline
            \multicolumn{2}{c}{Final Population} \\ \hline
            chromosome 1 &2  4  7  6  8  10  9  1  5  3 \\ \hline
            chromosome 2 & 4  10  1  8  9  2  6  7  5  3 \\ \hline
            chromosome 3 & 9  5  7  10  2  6  4  3  8  1 \\ \hline
            chromosome 4 &  6  4  3  10  9  5  2  7  1  8 \\ \hline
            chromosome 5 &  3  7  6  2  10  1  9  8  5  4 \\ \hline
            chromosome 6 &2  1  8  9  10  3  7  6  4  5 \\ \hline
            chromosome 7 &  9  10  6  4  1  3  7  2  5  8 \\ \hline
          chromosome 8 &  10  3  9  6  7  1  4  5  2  8 \\ \hline
          chromosome 9 &  3  7  5  4  10  2  1  6  9  8 \\ \hline
          chromosome 10 &  4  2  1  3  5  7  10  6  9  8 \\ \hline
          chromosome 11 &  2  4  7  6  8  1  3  5  9  10 \\ \hline
          chromosome 12 &  4  10  1  8  9  2  3  5  6  7 \\ \hline
          chromosome 13 &  9  5  7  10  2  1  3  4  6  8 \\ \hline
          chromosome 14 &  6  4  3  10  9  1  2  5  7  8 \\ \hline
          chromosome 15 &   3  7  6  2  10  1  4  5  8  9 \\ \hline
          chromosome 16 &  2  1  8  9  10  3  4  5  6  7 \\ \hline
          chromosome 17 &  9  10  6  4  1  2  3  5  7  8 \\ \hline
          chromosome 18 &  10  3  9  6  7  1  2  4  5  8 \\ \hline
            chromosome 19 &  3  7  5  4  10  1  2  6  8  9 \\ \hline
            chromosome 20 & 4  2  1  3  5  6  7  8  9  10 \\ \hline
\end{tabular}
\caption{Experimental results 2 Mutation Step}
\end{table}



\textsf{Experimental results 2 of Mutation step:}\\
    \colorbox{blue!30}{\textsf{     No. of chromosomes is: 20}}\\
    \colorbox{blue!30}{\textsf{     length of chromosome is: 10}}\\
    \colorbox{blue!30}{\textsf{     CrossOver Operator:One Point and Crossing}}\\
    \colorbox{blue!30}{\textsf{     Keeps best parent to next Population: false}}\\
    \colorbox{blue!30}{\textsf{     Two random Postions(2 bits of chromosome): 3 , 5}}\\
    \begin{table}[H]
        \centering
        \begin{tabular}{{ l l }}\hline
            \multicolumn{2}{c}{Final Population} \\ \hline
            chromosome 1 &2  4  7  6  8  10  9  1  5  3 \\ \hline
            chromosome 2 & 4  10  1  8  9  2  6  7  5  3 \\ \hline
            chromosome 3 & 9  5  7  10  2  6  4  3  8  1 \\ \hline
            chromosome 4 &  6  4  3  10  9  5  2  7  1  8 \\ \hline
            chromosome 5 &  3  7  6  2  10  1  9  8  5  4 \\ \hline
            chromosome 6 &2  1  8  9  10  3  7  6  4  5 \\ \hline
            chromosome 7 &  9  10  6  4  1  3  7  2  5  8 \\ \hline
          chromosome 8 &  10  3  9  6  7  1  4  5  2  8 \\ \hline
          chromosome 9 &  3  7  5  4  10  2  1  6  9  8 \\ \hline
          chromosome 10 &  4  2  1  3  5  7  10  6  9  8 \\ \hline
          chromosome 11 &  2  4  7  6  8  1  3  5  9  10 \\ \hline
          chromosome 12 &  4  10  1  8  9  2  3  5  6  7 \\ \hline
          chromosome 13 &  9  5  7  10  2  1  3  4  6  8 \\ \hline
          chromosome 14 &  6  4  3  10  9  1  2  5  7  8 \\ \hline
          chromosome 15 &   3  7  6  2  10  1  4  5  8  9 \\ \hline
          chromosome 16 &  2  1  8  9  10  3  4  5  6  7 \\ \hline
          chromosome 17 &  9  10  6  4  1  2  3  5  7  8 \\ \hline
          chromosome 18 &  10  3  9  6  7  1  2  4  5  8 \\ \hline
            chromosome 19 &  3  7  5  4  10  1  2  6  8  9 \\ \hline
            chromosome 20 & 4  2  1  3  5  6  7  8  9  10 \\ \hline
\end{tabular}
\caption{Experimental results 2 Mutation Step}
\end{table}



\newpage


\section{experimentation of Substitution Cipher}
\subsection{experimentation of population step}
\subsection{experimentation of Substitution Cipher step}
\subsection{experimentation of Fitness step}
\subsection{experimentation of Selection step}
\subsection{experimentation of CrossOver step}
\subsection{experimentation of Mutation step}