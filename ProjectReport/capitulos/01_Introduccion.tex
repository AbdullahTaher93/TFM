\chapter{Introduction}

Cryptanalysis is the technique of deriving the original message from the cipher text without any prior knowledge of secret key or the erivation of key from the cipher text. A general technique for cryptanalysis, applied to all cryptographic algos is to try all the possible keys until the correct key is matched, it is known as exhaustive key search. 

With every passing day, the computing ability of hardware is increasing manifold; therefore it becomes necessary to use long keys for avoiding exhaustive key search. All the other attacks applied to stream ciphers are compared to exhaustive key search in terms of data and memory complexity and if its complexity is less than exhaustive key search, then only these are considered as successful. 

A symmetric key cipher, especially a stream cipher is assumed secure, if the computational capability required for breaking the cipher by
best-known attack is greater than or equal to exhaustive key search. 
Cryptanalysis is the science of making encrypted data unencrypted use convert cipher text to plaintext because cryptanalysis used to convert
plaintext to cipher text and used cryptanalysis Return to plaintext or clear text or original text cryptanalysis is used to break codes by finding
weaknesses. There are many techniques used in the cryptanalysis. This project used the genetics algorithm \cite{bokhari2012cryptanalysis}.

The genetic algorithm is a search algorithm based on the mechanics of natural selection and natural genetics. The genetic algorithm belongs to
 the family of evolutionary algorithms, along with genetic programming,
Evolution strategies and evolutionary programming. The set of operators
usually consists of mutation, crossover and selection \cite{mangano1995genetic}.

\newpage
A genetic
algorithm has proven to be reliable and powerful optimization technique in
a wide variety of applications. It can be applied to both texts and images.
Genetic algorithm is secure since it does not utilize the natural numbers
directly. The genetic algorithm used for generating keys that it should be
good in terms of coefficient of autocorrelation \cite{sindhuja2014symmetric}.

Apply the technique of
genetic algorithms to the problem of finding the key to a particular
Transposition Cipher. Since Genetic Algorithms are primarily used to
efficiently search a large problem space we thought they would be ideally
suited for searching the large key space \cite{brownbridge2007decrypting}.

\section{aim of the project}
The main purpose of the project is examine the possible applications of genetic
algorithms in cryptology,with the emphasis of the research being in the application of a genetic algorithm in the  which explores the
plaintext from cipher text based on genetic algorithm (GA) which is used
for suggesting decryption key. The genetic algorithm (GA) is a search tool
to insure high probability of finding a solution by decreasing the amount
of time in the key space searching.
this project covers two ciphers type: Transposition Cipher and Substitution Cipher. 
We present a review of Transposition Cipher ,Substitution Cipher, English Letter Frequencies and genetic algorithm in chapters ***** which provides
enough background to understand the techniques applied and to assess the
usefulness of the results obtained.
\section{project organization}
In additional to chapter one (the introduction), this project is
organized into three other chapters:
-Chapter * \textbf{" Theoretical Background"} this chapter introduces terms,
definitions, and descriptions that are used frequently throughout this
chapter. These terms focus on Cryptanalysis, Transposition Cipher, and
description Genetic algorithm (GA).
Chapter Three \textbf{" Design and Implementation: Cryptanalysis based on
Genetic Algorithm"} this chapter shows the design of the project and
explains the algorithms that support this work.
Chapter Four \textbf{"Conclusions and Future Work"} this chapter concludes
this project and finding the outline future works in the area of
Cryptanalysis.
\newpage